\chapter{Conclusion}
This thesis showed that it is possible to use a standard compliant DTN
implementation on top of a IEEE 802.15.4 network. That was not obvious from the
beginning and can be seen as a success. We found and corrected problems in the
driver and Linux kernel stack for IEEE 802.15.4 and wrote a convergence layer to
adapt IBR-DTN to it. The evaluation was used for performance measurement and
showed us the actual performance as well as areas for improvements.

Being the first project implementing standard compliant DTN over a IEEE 802.15.4
link allowed us to proof that both technologies can work together. This opens up
a complete new field of research. Using delay tolerant networking in low-power
wireless sensor networks allows multi-hop transmission of sensor data following
the store and forward principle of DTN. Nodes must no longer be in range to
transmit data but can send it out to all other nodes in range which will forward
the data until they reach the destination. The win is a greater flexibility for
sensor node placement and moving sensors.

\section{Future Work}
The new convergence layer supports sending and receiving for bundles up to a
payload of 40 bytes. That is the biggest limitation of the current work. To
improve this situation different approaches can be taken. One of the newest
additions in IBR-DTN is header compression. This technique makes it possible to
shrink the header by using numerical addresses and avoid the dictionary to be
transmitted. Given that the current ratio in a bundle over IEEE 802.15.4 is two
thirds for the header and only one third for the actual payload that would indeed
improve the situation. To solve the problem for larger bundles a fragmentation
within the convergence layer could be used to split the bundles into smaller
chunks, transmit and then reassemble them within the convergence layer on the
receiving side. Such a mechanism is already described in the TCP convergence
layer draft~\cite{tcp-clayer-draft} and implemented in IBR-DTN. In the future
work this could be adapted to the IEEE 802.15.4 convergence layer to fully
solve the problem of the restricted payload length.

The evaluation also revealed some initial performance statistics which can be
used to improve the current implementation. One of the most interesting points
is that the raw IEEE 802.15.4 data throughput performance, without any DTN
involved, is already way below the theoretical maximum of 250 kbps. With only
4600 bps it reaches only 1.8 percent of the theoretical maximum. Partly this
comes from a delay added in the test application to allow a reliable test
environment still this explains not everything. Future research on this topic
should correct all problems in the driver and kernel stack to allow reliable
data transfer without any delay added in the application. This will help to
pinpoint other performance bottlenecks in the driver and Linux stack as well as
in IBR-DTN.
