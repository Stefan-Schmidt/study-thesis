\chapter{Introduction}
\cite{RFC5050}

\section{IEEE 802.15.4}

Driven by an IEEE working group the 802.15.4 standard provides the physical and
mac access control layer for so called low-rate wireless personal area network
(LR-WPAN). The emphasis is on low cost devices for communication without
infrastructure in a nearby environment. The communication range would be up to
10m and offers nowadays a transfer rate up to 250 kbit/s. The standard specifies
three possible frequency bands to operate in. The usage of some may be
restricted in different countries but one band is available worldwide. Other
features of 802.15.4 includes collision avoidance through CSMA/CA, build in
support for secure communication through cryptography, power management through
link quality control and energy detection as well as reserved time slots for
real-time operations.

As already covered the standard does only cover the two lowest layer of a
protocol stack. Supplement it to a fully functional networking stack is the aim
of different other specifications. The most famous would be ZigBee. With 6LowPAN
there is also work underway to link LR-WPAN together with standard internet
protocols like IPv6.

Specified for an infrastructure-less network the topologies may be star or
Peer-to-Peer based as well a composition of both. Two different device types are
allowed. full-function device (FFD) and reduced-function device (RFD). The RFD
is a really simple device which can only connect to one FFD at a time. Therefor
it can only act as a leaf in all described topologies. In contrast the FFD is
able to act as a coordinator to span up a whole PAN and rely messages to other
nodes. At least one coordinator is needed in every network. One thing to keep in
mind is that routing is not covered by the standard. To rely messages over
different hops the supplement upper layers need to take care of this.

\section{DTN}
