\chapter{Base System}
This chapter describes the basic software system we are using on the Imote2
platform. Ranging from the bootloader through the Linux kernel up to the
userland software stack. Our DTN implementation, IBR-DTN, is excluded here as we
describe them in detail in chapters \ref{ibr-dtn} and \ref{802154layer}.

\section{Bootstraping}
The Imote2 comes with TinyOS pre-installed. No bootloader or similar mechanism
is installed to allow reflashing of the device. As a last resort there is JTAG
available on the interface board which we used  with OpenOCD
\footnote{http://openocd.berlios.de/} and a
JTAGKeyTiny adapter to bootstrap our Linux system. Blob was used as bootloader to
load the kernel image from flash and set the right parameters to boot into the
userland root file system. Detailed steps how the image can be flashed into the
device via JTAG can be found in appendix \ref{annexopenocd}.

\section{Linux Kernel}
A recent Linux kernel (2.6.33) is used on the device. The basic support for the
PXA271 and the Imote2 board in particular was already available in this kernel.
The support included the usage of the on board flash chip for storage as well as
the USB client to act as a USB ethernet device showing up at the development host.
Sadly the CPU frequency scaling support for the PXA platform does not work
correctly. Once enabled the access to the flash did no longer work.
Solving of this was out of scope for this thesis but may be interesting to do for a
power consumption evaluation of the Imote2.

\section{CC2420 Driver and ieee802154 stack}
Different drivers have been available for the CC2420. An older TosMac driver and
a newer driver which used the ieee802154 networking stack from the Linux kernel.
We decided to go with the latter one to have something that could go into the
mainline kernel and therefore provide us with an easier long term use, API
stability and maintenance.

Like the IEEE 802.15.4 protocol the Linux stack is divided into two layers. The
physical layer which is called ieee802154 and the mac layer which is called
mac802154. While the former is already merged into the mainline kernel the latter
is still being developed and lives inside the kernel repository of the
linux-zigbee \footnote{http://sourceforge.net/apps/trac/linux-zigbee/} project.
This repository follows the mainline kernel very closely
and we based our work on it to submit fixes and enhancements directly to the
driver maintainer.

The CC2420 driver based on the new Linux stack was submitted right in time for
our research. Some debugging and bug fixing had to be applied to this driver
before it was fully functional. The first problem was hidden in the hardware
support for the different IEEE 802.15.4 related tasks which include  header
preparation in hardware. That way it is not needed to
completely assemble or disassemble a packet in the driver and the stack. While
some parts are static enough to be set depending on some configuration registers
the length of an outgoing packet data unit (PDU) needs to be set from the driver
as it is the only piece that knows about the upcoming data length. Due to this
the first byte that has to enter the data FIFO of the CC2420 has to be the
length byte of the PDU. The initial driver missed that and this resulted in
wrong packet length and broken packet headers as the first byte of the header
was interpreted as the length byte. With this problem out of the way sending
packets worked like it should.

On the receiving side another subtle problem had to be found and eliminated
before the driver could be fully used. Another hardware accelerated task is the
CRC checking of an incoming packet. The 16bit CRC sits in the last two bytes of
an incoming packet. If enabled, the hardware checks for the correct CRC and
then replaces these two bytes with a flag about the correctness of the CRC, the
RSSI value and an average correlation which can be used for LQI calculation
later on, before pushing it up to the driver. The driver now filled in the RSSI
value as signalling information for the stack and made sure to cut one byte
of the packet before pushing it as data into the stack. The second byte was
still attached and this resulted in failing length checks for MAC commands
inside the stack. Cutting of the second byte as well allowed us to receive
and parse incoming packets and get bi-directional communication working. All
fixes are sitting together with the cc2420 driver in the linux-zigbee tree to
get merged with the mac802154 layer once it is mature enough to hit the
mainline Linux kernel.

\section{OpenEmbedded base system}
On top of the Linux kernel we needed a small Linux root file system which
contains all userland components that are in use to get data from the kernel
into IBR-DTN as well as development and debugging tools.

We have chosen OpenEmbedded as the build system here to provide us with a working
kernel image and matching root filesystem we could flash onto the device. In the
development phase OpenEmbedded was also used to cross compile the code for the
target architecture. Some more information about OpenEmbedded and our use on the
Imote2 could be found in appendix \ref{annexoe}.
