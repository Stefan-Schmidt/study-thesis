\chapter{IBR-DTN}
\label{ibr-dtn}
\section{IBR-DTN}

IBR-DTN is an implementation of the DTN bundle protocol~\cite{RFC5050}. It is
designed and implemented with efficiency in mind. Efficient memory usage is critical
for use in embedded systems like wireless routers or smartphones. Another design
goal is the interoperability with the DTN2 reference implementation by the Delay
Tolerant Networking Research Group (DTNRG). In contrast to the reference
implementation it can cope with only 4 MByte of memory. That way it can be run
on wireless routers or similar constrained devices.

It features a UDP, TCP and a HTTP convergence layer. All three are targeted at
compliance to the respective internet drafts. The work done for this thesis lays
the groundwork for the forthcoming convergence layer in IBR-DTN. The details and
shortcomings of our work are discussed in chapter~\ref{802154layer}. All these
convergence layers are handled by the Convergence Layer Manager as part of the
modular design of IBR-DTN.

Another module handles the storage of bundles either in memory or on persistent
storage. This allows flexible handling of different kinds of scenarios. With
persistent storage it is for example possible to forward a bundle received earlier
even if the device had a power outage. To know which other nodes are reachable
IBR-DTN offers two mechanisms. The first one is discovery which can be based on
the UDP or TCP convergence layer as well as the IP neighbour discovery draft for
DTN. The second way is to set static routes for the nodes inside the
configuration file of IBR-DTN. This maps the name of a DTN node to its address
and maybe some needed routing. Next to the default routing strategy there is
also an implementation of epidemic routing with bloomfilter.

Once a new node is discovered the bundles stored for it are forwarded immediately.
Some small applications that allow comfortable testing and debugging are also
included in the suite: Dtnping for round trip time measurement as well as simple
bundle send and receive tools.

\section{DTN Convergence Layer}
The role of a convergence layer in DTN is to encapsulate the details of the
transport protocol from the rest of the bundle protocol. If the DTN core is in
need to send out a bundle it uses the \emph{queue} method of the convergence
layer to pass the bundle data and destination node informations to the
convergence layer.

All details about the bundle transmission are handled within the convergence
layer and its helper functions. How complex the convergence layer needs to be
depends on the transport protocol it implements. In general it can be split into
three parts: establishing a connection, transfer the data and closing a
connection.

The receiving path the convergence layer listens for incoming data and up on
complete receive sends the bundle into the DTN core for further processing. All
details about possible packet reassembling or fragmentation are again hidden
from the DTN core by the convergence layer.

\section{Other DTN Implementations}
As briefly mentioned above there are other implementations of the bundle
protocol available. ION~\footnote{https://ion.ocp.ohiou.edu/} is, like IBR-DTN,
a Linux/Unix implementation. DASM~\cite{dasm} is an implementation for Symbian.
DTNLite~\cite{dtnlite} and ContikiDTN~\cite{contikidtn} are written for TinyOS and
Contiki respectively. Those are the ones that are the closest to our work we
have done here. We discuss them in section~\ref{relatedwork}.
