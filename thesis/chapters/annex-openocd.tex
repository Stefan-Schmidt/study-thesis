\parindent=0pt                   % erzeugt bei einem Absatz eine
\parskip=6pt plus 3pt            % Leerzeile (kein Einruecken)

\label{annexopenocd}

Use OpenOCD to do the initial blob install:

The iMote2 is shipped with TinyOS by default. There is now bootloader installed
that would allow you to flash another OS so the only way to get it bootstrapped
was to use the JTAG interface.

We used OpenOCD for this task. Make sure you have at least OpenOCD version 0.4.0
and libftdi version 0.17 installed.

The OS e are going to install consists of the major parts. The bootloader blob, the
Linux kernel and a root file system with the userland software components. While
we are preparing the kernel and the root filesystem ourselfs we use a
precompiled version of the blob bootloader from:
http://sourceforge.net/projects/imote2-linux/files/

The used JTAG adapter is a JTAGkey Tiny from Amontec. OpenOCD already offers
configurations for the JTAG adapter as well as the iMOte2 board. Starting
OpenOCD as following uses the right configuration files:

sudo openocd -f /usr/share/openocd/scripts/interface/jtagkey.cfg -f \\
/usr/share/openocd/scripts/board/crossbow\_tech\_imote2.cfg

Now you ca connect to the OpenOCD telnet interface on port 4444 with:

\begin{verbatim}
telnet localhost 4444
\end{verbatim}

First we need to make sure that the iMote2 is halted. In this state we are at PC
0 and no code got exectued so far. Afterwards disbale the flash protection and
erase the complete flash. The last three steps are writing the bootloader, the
linux kernel and the rootfs at the given offsets into the flash. The last step
will take a long time as it writes 30 Mbyte into the flash.

\begin{verbatim}
# reset halt
# flash protect 0 0 last off
# flash erase_sector 0 0 258
# flash write_image blob-im2
# flash write_image 2.6.34-rc2-zImage 0x00040000 bin
# flash write_image console-image-imote2.jffs2 0x00240000 bin
\end{verbatim}
