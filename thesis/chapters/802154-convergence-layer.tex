\chapter{IEEE 802.15.4 Convergence layer}
\label{802154layer}
\section{Interface with the Linux ieee802154 Stack}

Like the convergence layers for UDP or TCP the job of the IEEE 802.15.4
convergence layer is to mediate between the Linux kernel provided interface
to the ieee802514 stack and the IBR-DTN core. It has no knowledge about the
bundle protocol or other DTN specific internals but provides only classes and
functions to find a node, address it and transfer the raw data.

The communication between the IEEE 802.15.4 convergence layer of IBR-DTN and the
Linux ieee802154 stack is done over two separate channels. One is used for
signalling and configuration and is handled over the kernel netlink mechanism.
The second is used for the actual data transport and is exposed as standard
socket interface. Listing \ref{lstsocketapi} shows the usage of the socket interface.
As domain AF\_IEEE802154 is used and a struct sockaddr\_ieee802154 fulfills the
need for an adjustment of the addressing scheme as well as the usage of a PAN
ID. The type is the known SOCK\_DGRAM and the default protocol is used. In the
convergence layer we use the socket interface to transmit the data to the actual
peer. The functionlaity is wrapped into a UnicastSocketLOWPAN class.

\begin{lstlisting}[caption= ieee802154 socket interface, label=lstsocketapi]
        int sd;
        struct sockaddr_ieee802154 a;

        sd = socket(PF_IEEE802154, SOCK_DGRAM, 0);

        a.family = AF_IEEE802154;
        a.addr.addr_type = IEEE802154_ADDR_SHORT;
        a.addr.pan_id = 0x0780;

	a.addr.short_addr = 0x0001;
	bind(sd, (struct sockaddr *)&a, sizeof(a));

        a.addr.short_addr = 0x8001;
	connect(sd, (struct sockaddr *)&a, sizeof(a));
\end{lstlisting}

Over netlink it is possible to send different commands and a set attributes to the
ieee802154 stack. The attributes contain among others informations about the
address of the interface, its index, PAN ID, channel and more. The commands
allow to get and set the different attributes, trigger a association, request a
scan and others. The given functionality is used to implement the basic
utilities for configuration as well as a PAN coordinator bundled together in
the lowpan-tools package. In IBR-DTN we only use a small subset of this
functionality. We use the given interface name to ask for its short address when
binding to the socket to receive incoming data. In the future we could also ask for
the PAN ID and the hardware address to be more flexible.

For the basic network setup we rely on the lowpan-tools utilities. With
\emph{izcoordinator} we set up a PAN coordinator on one
node while we use \emph{iz} to associate to this PAN from the other node. The PAN ID is
given to both tools on the command line and the coordinator hands the
associating node a short address. With this basic configuration, which can
somewhat be compared to a DHCP server and client, we start dtnd and are ready to
transmit and receive bundles.

\section{Configuration}

To make IBR-DTN work correctly with our IEEE 802.15.4 hardware we need to configure
it and teach it about the interface name it has to use as well as that we are
looking for a zigbee connection. Lowpan is just the internal name for the
IEEE 802.15.4 convergence layer inside IBR-DTN. The PAN ID needs to be filled into the
port configuration option. Filled in values need to be in decimal notation.

Listing \ref{lstconfig} also shows the configuration for static routes to other
nodes.

\begin{lstlisting}[caption= dtnd example configuration, label=lstconfig]
cal_uri = dtn://1.dtn
net_interfaces = lan0
net_lan0_type = lowpan
net_lan0_interface = wpan0
net_lan0_port = 1920		#0x780

routing = default

# Static connections
static1_address = 32769         #0x8001
static1_port = 1920		#0x780
static1_uri = dtn://2.dtn
static1_proto = zigbee
\end{lstlisting}

\section{Addressing}

The specification for Low-Rate Wireless Personal Area Networks specifies two
types of addresses. All devices have an unique 64 bit extended address.
Additional they can have a 16 bit short address that can be different depending
on the PAN. The short address is supplied by the PAN coordinator. For now we
only support the short address type in IBR-DTN. Support for extended addresses
are planned for the future. As well as the discovery of other nodes over a
broadcast mechanism. Such a braodcast could be done by sending it to the 0xffff
address with the correct PAN ID to reach all nodes within the PAN or with 0xffff
as PAN ID to reach all nodes in range.

\section{Fragmentation}

The maximal packet size in IEEE 802.15.4 is about 128 byte. Deducting the header
and other needed data we can transport about 115 byte of actual payload in a
packet. This payload size is way smaller to what other IBR-DTN convergence
layers have used before. They have been build around the IP/TCP/UDP family which
normally has a maximal transfer size of 1500 byte.

The header size of the bundle protocol is not fixed but variable depending on
the transported data and enabled options. It is as well possible that the header
size is greater then the 115 byte payload we could offer with IEEE 802.15.4. The
draft \cite{tcp-clayer-draft} for a TCP convergence layer protocol could help here as it
defines a data transmission divided into several segments. Even the bundle
header could be split into several segments this way and build together on the
receiving side.

\section{Related Work}
\label{relatedwork}

While DTN is getting more popular for research there are only a few research
projects then combine DTN with sensor network technologies like IEEE 802.15.4.
That is not surprising if one keeps in mind that the DTN architecture is is
complex and memory as well as bandwidth consuming. Both resources that are scare
on sensor network platforms.

Two projects are aiming to explore this area of research. One of them is the
TinyOS based DTNLite \cite{dtnlite}. It is loosely based on the DTN overlay
architecture. To deliver custody transfer over the overlay they use techniques
like asynchronous message delivery. The concept is targeted at data collection
applications which need to reach a central station for the collection over the
network. The applications must be written or re-written to be DTNLite aware.

ContikiDTN \cite{contikidtn} is, like DTNLite, written for a special purpose
embedded operating system in this case Contiki
\footnote{http://www.sics.se/contiki/}. While DTNLite only aims to work with
other DTNLite installations Contiki DTN aims for full standard compliance.
ContikiDTN is tied to only work with a TCP convergence layer. It was a design
choice that no other convergence layer can be plugged in. This restriction goes
down to the used µIP component of Contiki which offers the used protosockets
only work with TCP.

Both implementations have not been suitable for our work as they are either not
standard compliant or have a design restriction to only TCP. Furthermore both
are designed around a special purpose operating system while our aim was to work
on the general purpose operating system Linux.
