\chapter{802.15.4 Convergence layer}
\label{802154layer}
\section{Interface with the linux ieee802154 Stack}
\section{Addressing}
\section{Fragmentation}
\section{Related Work}

While DTN is getting more popular for research there are only a few research
projects then combine DTN with sensor network technologies like IEEE 802.15.4.
That is not surprising if one keeps in mind that the DTN architecture is is
complex and memory as well as bandwidth consuming. Both resources that are scare
on sensor network platforms.

Two projects are aiming to explore this area of research. One of them is the
TinyOS based DTNLite \cite{dtnlite}. It is loosely based on the DTN overlay
architecture. To deliver custody transfer  over the overlay they use techniques
like asynchronous message delivery. The concept is targeted at data collection
applications which need to reach a central station for the collection over the
network.

Contiki DTN

