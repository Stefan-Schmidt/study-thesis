\chapter{OpenEmbedded}
\label{annexoe}

OpenEmbedded \footnote{http://www.openembedded.org/} is a build framwork for
embedded Linux. It provides the meta-data which incorporates the knowledge how
to fetch, build, install, package and more for over 5000 Open Source projects.

Adding support for the Imote2 hardware to it did only required us to add a
\emph{machine configuration} which describes the hardware and let OpenEmbedded
know about it. Below is this machine configuration we added:

\begin{verbatim}
#@TYPE: Machine
#@Name: Crossbow iMote2
#@DESCRIPTION: Machine configuration for Crossbow iMote 2
TARGET_ARCH = "arm"
PREFERRED_PROVIDER_virtual/kernel = "linux"
PACKAGE_EXTRA_ARCHS = " iwmmxt"
KERNEL_IMAGETYPE = "zImage"
IMAGE_FSTYPE += "jffs2"
EXTRA_IMAGECMD_jffs2 = "--l --pad=0x01DC0000 --eraseblock=0x20000"
CMDLINE="root=/dev/mtdblock2 rootfstype=jffs2 console=ttyS2,115200"
SERIAL_CONSOLE = "115200 ttyS2"
require conf/machine/include/tune-xscale.inc
ROOT_FLASH_SIZE ?= "30"
MACHINE_FEATURES = "kernel26 usbgadget alsa iwmmxt"
\end{verbatim}

We also added the needed meta-data for the Linux kernel used on our Imote2 and
the build description for IBR-DTN. In a final step we put all this together in
an image that could be build with the \emph{bitbake} application. Bitbake uses
the meta-data provided by OpenEmbedded to actually build the software and the
resulting firmware and kernel images.

After this we were able to create our complete firmware image by just executing
bitbake like this:

\begin{verbatim}
bitbake imote2-image
\end{verbatim}

All our changes are already included in the OpenEmbedded source code repository
and could be used from there. The latest git revision we used for our work was
6fc1c96131d4c4afabe815148f8e55656fb7f402. If any newer version have problems it
is possible to revert back to this known good revision for future work.
