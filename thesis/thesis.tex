\documentclass[11pt,twoside]{scrbook}

\usepackage{a4}
\usepackage{times}
\usepackage{graphicx}
%\usepackage{verbatimfiles}

\usepackage[outline,light]{draftcopy}

% So wird's gemacht, wenn in den Text EPS-Bilder eingefuegt werden sollen!
% \begin{figure}[htb]
% \centerline{\includegraphics[width=\textwidth]{bild1.eps}}
% \caption{\label{bild1}Title of the picture.}
% \end{figure}

\begin{document}

%Deckblatt erzeugen
\thispagestyle{headings}
\title{Delay Tolerant Networking on Embedded Linux Hardware}
\author{Stefan Schmidt}

\maketitle
\setcounter{page}{1}
\pagenumbering{roman}

\thispagestyle{headings}
\begin{titlepage}
	\let\footnotesize\small
	\let\footnoterule\relax
	\null
	\vfil
	\vskip 60pt
	\begin{center}
		{\LARGE
			{\Large Technische Universität Braunschweig}\\
			Institut für Betriebssysteme und
				Rechnerverbund\\[2cm]
			{\Large Studienarbeit}\\ [2cm]
			Delay Tolerant Networking on Embedded Linux Hardware
			\par}%
		\vskip 6em
		{\large \lineskip .75em
		\begin{tabular}[t]{c}
			{\Large von}\\[.5em]
			{\Large cand. inform.\ Stefan Schmidt}\\[7em]
			{\bf Aufgabenstellung und Betreuung:}\\[.5em]
			Prof.\ Dr.\ L.\ Wolf.  und
			Dipl.-Inform.\ W.\ Pöttner\\
		\end{tabular}
		\par}%
		\vfill
		{\large
			Braunschweig, den \today
			\par}%
	\end{center}
	\par
	% thanks
	\vfil
	\null
\end{titlepage}

~\newpage

% Erklaerung
\vspace*{7cm}
\centerline{\bf Erklärung}

\vspace*{1cm}
Ich versichere, die vorliegende Arbeit selbstständig und nur unter Benutzung
der angegebenen Hilfsmittel angefertigt zu haben.

\vspace*{3cm}
%\centerline{Braunschweig, den \today \hfil \hfil Unterschrift}
Braunschweig, den \today

\pagestyle{headings}
\cleardoublepage

\centerline{\bf Abstract}

This example abstract is indeed very abstract.

\cleardoublepage

\centerline{\bf Aufgabenstellung}
Ziel dieser Arbeit ist es, eine lauffähige Version der IBR DTN Implementierung
für die iMote 2 Hardwareplattform zu erstellen, welche die Funkschnittstelle
CC2420 zum Bündelaustausch mit anderen iMote 2 Knoten verwendet.

Im Rahmen dieser Arbeit ist deshalb zunächst ein Überblick über den
zugrundeliegenden Standard IEEE 802.15.4 zu geben. Weiterhin ist ein Überblick
über verwandte Arbeiten zum Thema Delay Tolerant Networking auf IEEE 802.15.4
basierter Hardware zu erarbeiten.

Anschließend muss eine Entwicklungsumgebung auf Basis von OpenEmbedded
aufgesetzt werden, um schließlich IBR DTN für die Zielplattform übersetzen zu
können. Weiterhin muss ein geeigneter Treiber für den Chipcon CC2420 entwickelt
sowie eine passende Schnittstelle spezifiziert und entwickelt werden. Sollte
sich einer der existierenden Treiber als nutzbringend erweisen, so kann dieser
entsprechend den Anforderungen angepasst werden. Im letzten Schritt muss nun
eine Anpassungsschicht (Convergence Layer) implementiert werden, mit deren Hilfe
IBR DTN die Funkschnittstelle zum Datenaustausch zwischen Knoten nutzen kann.
Dieser Convergence Layer soll dabei mindestens den Bündelaustausch zwischen
zwei Knoten sowie die Erkennung von Knoten in der Umgebung ermöglichen.

Die neue Implementierung sowie die Leistungsfähigkeit der Kombination aus
iMote 2, Linux, CC2420 und IBR DTN ist zu evaluieren. Dabei soll insbesondere
der erzielbare Durchsatz sowie der protokollbedingte Overhead ermittelt werden.
Engpässe in der Leistungsfähigkeit sollen nach Möglichkeit der verantwortlichen
Komponente zugeordnet werden, um Ansatzpunkte für zukünftige Optimierungen
aufzuzeigen. Die entwickelte Software ist weiterhin auf korrekte Funktion zu
testen. Die Testfälle sollten mindestens das Auffinden von anderen Knoten, den
Datenaustausch zwischen Knoten, die Erneutübertragung verloren gegangener Frames
sowie die Fragmentierung und Wiederherstellung von Bündeln abdecken.

\tableofcontents
\cleardoublepage
\listoffigures
\cleardoublepage
\listoftables
\cleardoublepage

%\parindent=0pt                   % erzeugt bei einem Absatz eine
%\parskip=6pt plus 3pt            % Leerzeile (kein Einruecken)

\setcounter{page}{0}

\pagestyle{headings}
\pagenumbering{arabic}

\chapter{Introduction}
\cite{RFC5050}


\include{chapters/02-imote2-hardware}

\include{chapters/03-ibr-dtn-implementation}

\chapter{802.15.4 Convergence layer}


\include{chapters/05-evaluation}

\include{chapters/06-conclusion}

\addcontentsline{toc}{chapter}{Literature}
\bibliographystyle{unsrt}
\bibliography{thesis}

\begin{appendix}

Use OpenOCD to do the initial blob install:

The iMotes are shipped with something like TinyOS by default. I wasn't able to
find a bootloader on it so the only way to get it bootstrapped was to use the
JTAG interface.

\begin{verbatim}
Get OpenOCD (v0.4.0) and libftdi (v0.17)
Download the pre-compiled blob from: http://sourceforge.net/projects/imote2-linux/files/
(As root or with fixed US permissions) sudo openocd -f /usr/share/openocd/scripts/interface/jtagkey.cfg -f /usr/share/openocd/scripts/board/crossbow_tech_imote2.cfg
On another console: telnet localhost 4444
# reset halt
# flash protect 0 0 last off
# flash erase_sector 0 0 258
# flash write_image blob-im2
# flash write_image 2.6.34-rc2-zImage 0x00040000 bin
# flash write_image console-image-imote2.jffs2 0x00240000 bin
\end{verbatim}

\chapter{OpenEmbedded}
\label{annexoe}

OpenEmbedded~\footnote{http://www.openembedded.org/} is a build framework for
embedded Linux. It provides the meta-data which incorporates the knowledge how
to fetch, build, install, package and more for over 5000 Open Source projects.

Adding support for the Imote2 hardware to it did only required us to add a
\emph{machine configuration} which describes the hardware and let OpenEmbedded
know about it. Below is this machine configuration we added:

\begin{verbatim}
#@TYPE: Machine
#@Name: Crossbow iMote2
#@DESCRIPTION: Machine configuration for Crossbow iMote 2
TARGET_ARCH = "arm"
PREFERRED_PROVIDER_virtual/kernel = "linux"
PACKAGE_EXTRA_ARCHS = " iwmmxt"
KERNEL_IMAGETYPE = "zImage"
IMAGE_FSTYPE += "jffs2"
EXTRA_IMAGECMD_jffs2 = "--l --pad=0x01DC0000 --eraseblock=0x20000"
CMDLINE="root=/dev/mtdblock2 rootfstype=jffs2 console=ttyS2,115200"
SERIAL_CONSOLE = "115200 ttyS2"
require conf/machine/include/tune-xscale.inc
ROOT_FLASH_SIZE ?= "30"
MACHINE_FEATURES = "kernel26 usbgadget alsa iwmmxt"
\end{verbatim}

We also added the needed meta-data for the Linux kernel used on our Imote2 and
the build description for IBR-DTN. In a final step we put all this together in
an image that could be build with the \emph{bitbake} application. BitBake uses
the meta-data provided by OpenEmbedded to actually build the software and the
resulting firmware and kernel images.

After this we were able to create our complete firmware image by just executing
BitBake like this:

\begin{verbatim}
bitbake imote2-image
\end{verbatim}

All our changes are already included in the OpenEmbedded source code repository
and could be used from there. The latest git revision we used for our work was
6fc1c96131d4c4afabe815148f8e55656fb7f402. If any newer version have problems it
is possible to revert back to this known good revision for future work.


\end{appendix}

\end{document}
