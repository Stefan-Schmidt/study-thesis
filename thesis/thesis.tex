\documentclass[11pt,twoside]{scrbook}

\usepackage{a4}
\usepackage{german}
\usepackage{times}
\usepackage{graphicx}
%\usepackage{verbatimfiles}

\usepackage[outline,light]{draftcopy}

% So wird's gemacht, wenn in den Text EPS-Bilder eingefuegt werden sollen!
% \begin{figure}[htb]
% \centerline{\includegraphics[width=\textwidth]{bild1.eps}}
% \caption{\label{bild1}Title of the picture.}
% \end{figure}


\begin{document}

%Deckblatt erzeugen
\thispagestyle{headings}
\title{Delay Tolerant Networking on Embedded Linux Hardware}
\author{Stefan Schmidt}

\maketitle
\setcounter{page}{1}
\pagenumbering{roman}

\thispagestyle{headings}
\begin{titlepage}
	\let\footnotesize\small
	\let\footnoterule\relax
	\null
	\vfil
	\vskip 60pt
	\begin{center}
		{\LARGE
			{\Large Technische Universit"at Braunschweig}\\
			Institut f"ur Betriebssysteme und
				Rechnerverbund\\[2cm]
			{\Large Studienarbeit}\\ [2cm]
			Delay Tolerant Networking on Embedded Linux Hardware
			\par}%
		\vskip 6em
		{\large \lineskip .75em
		\begin{tabular}[t]{c}
			{\Large von}\\[.5em]
			{\Large cand. inform.\ Stefan Schmidt}\\[7em]
			{\bf Aufgabenstellung und Betreuung:}\\[.5em]
			Prof.\ Dr.\ L.\ Wolf.  und
			Dipl.-Inform.\ W.\ P"ottner\\
		\end{tabular}
		\par}%
		\vfill 
		{\large
			Braunschweig, den \today
			\par}%
	\end{center}
	\par
	% thanks
	\vfil
	\null
\end{titlepage}

~\newpage

% Erklaerung
\vspace*{7cm}
\centerline{\bf Erkl"arung}

\vspace*{1cm}
Ich versichere, die vorliegende Arbeit selbstst"andig und nur unter Benutzung
der angegebenen Hilfsmittel angefertigt zu haben.

\vspace*{3cm}
%\centerline{Braunschweig, den \today \hfil \hfil Unterschrift}
Braunschweig, den \today 

\pagestyle{headings}
\cleardoublepage

\centerline{\bf Abstract}

This example abstract is indeed very abstract.

\cleardoublepage

\vspace*{7cm}
\centerline{[Hier wird sp"ater die Aufgabenstellung eingef"ugt.]}




\tableofcontents		% Inhaltsverzeichnis erzeugen
\cleardoublepage
\listoffigures			% Bilderverzeichnis erzeugen
\cleardoublepage
\listoftables			% Tabellenverzeichnis erzeugen
\cleardoublepage


%\parindent=0pt                   % erzeugt bei einem Absatz eine
%\parskip=6pt plus 3pt            % Leerzeile (kein Einruecken)

\setcounter{page}{0}

\pagestyle{headings}
\pagenumbering{arabic}

%\include{einleitung}		% Einleitung
\chapter{Einleitung}

Im IP Protokoll-Header wird das DS-Feld \cite{RFC2474} benutzt, um...

\cleardoublepage

%\include{kapitel2}	        % 2. Kapitel
%\cleardoublepage

%\include{kapitel3}		% 3. Kapitel
%\cleardoublepage
                                % usw.

%\include{ausblick}		% Zusammenfassung und Ausblick
\chapter{Zusammenfassung und Ausblick}
\cleardoublepage

\addcontentsline{toc}{chapter}{Literaturverzeichnis}
\bibliographystyle{unsrt}
\bibliography{thesis}

\begin{appendix}

%\include{anhang1}		% Anhang A
%\cleardoublepage
                                % usw.

\end{appendix}

\end{document}
