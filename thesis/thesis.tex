\documentclass[11pt,twoside]{scrbook}

\usepackage{a4}
\usepackage{times}
\usepackage{graphicx}
\usepackage{listings}
\usepackage[english,iso]{isodate}
\usepackage[utf8x]{inputenc}
\usepackage{tocloft}
\usepackage{pdfpages}
%\usepackage{verbatimfiles}

%\usepackage[outline,light]{draftcopy}

\usepackage[pdfauthor={Stefan Schmidt},pdftitle={Delay Tolerant Networking on
Embedded Linux Hardware},pdfkeywords={802.15.4, LowPAN, LR-WPAN, DTN, IBR-DTN,
linux, Imote2}]{hyperref}

% We want the caption position of listings at the bottom
\lstset{captionpos=b}

% So wird's gemacht, wenn in den Text EPS-Bilder eingefuegt werden sollen!
% \begin{figure}[htb]
% \centerline{\includegraphics[width=\textwidth]{bild1.eps}}
% \caption{\label{bild1}Title of the picture.}
% \end{figure}

\begin{document}

%Deckblatt erzeugen
\thispagestyle{headings}
\title{Delay Tolerant Networking on Embedded Linux Hardware}
\author{Stefan Schmidt}

\maketitle
\setcounter{page}{1}
\pagenumbering{roman}

\thispagestyle{headings}
\begin{titlepage}
	\let\footnotesize\small
	\let\footnoterule\relax
	\null
	\vfil
	\vskip 60pt
	\begin{center}
		{\LARGE
			{\Large Technische Universität Braunschweig}\\
			Institut für Betriebssysteme und
				Rechnerverbund\\[2cm]
			{\Large Studienarbeit}\\ [2cm]
			Delay Tolerant Networking on \\
			Embedded Linux Hardware
			\par}%
		\vskip 6em
		{\large \lineskip .75em
		\begin{tabular}[t]{c}
			{\Large von}\\[.5em]
			{\Large cand. inform.\ Stefan Schmidt}\\[7em]
			{\bf Aufgabenstellung und Betreuung:}\\[.5em]
			Prof.\ Dr.\ L.\ Wolf.  und
			Dipl.-Inform.\ W.\ Pöttner\\
		\end{tabular}
		\par}%
		\vfill
		{\large
			Braunschweig, den \today
			\par}%
	\end{center}
	\par
	% thanks
	\vfil
	\null
\end{titlepage}

~\newpage

% Erklaerung
\vspace*{7cm}
\centerline{\bf Erklärung}

\vspace*{1cm}
Ich versichere, die vorliegende Arbeit selbstständig und nur unter Benutzung
der angegebenen Hilfsmittel angefertigt zu haben.

\vspace*{3cm}
%\centerline{Braunschweig, den \today \hfil \hfil Unterschrift}
Braunschweig, den \today

\pagestyle{headings}
\cleardoublepage

\centerline{\bf Abstract}

Implementing and evaluating an IEEE 802.15.4 convergence layer for Delay
Tolerant Network is the aim of this thesis. We discuss how we prepare the
underlying Linux system, correct problems in the radio driver and
implement the convergence layer for IBR-DTN. Based on this implementations we
evaluate data throughput as well as packetloss and round trip time for either
the raw IEEE 802.15.4 network or with the DTN layer on top. The potential
bottlenecks revealed by the evaluation are discussed and future improvements
are suggested.

\cleardoublepage

\includepdf[pages={1,2}]{../ibr_aufgabenstellung.pdf}

\cleardoublepage
\tableofcontents
\cleardoublepage
\listoffigures
%\cleardoublepage
\listoftables
%\cleardoublepage

%\parindent=0pt                   % erzeugt bei einem Absatz eine
%\parskip=6pt plus 3pt            % Leerzeile (kein Einruecken)

\setcounter{page}{0}

\pagestyle{headings}
\pagenumbering{arabic}

\chapter{Introduction}

\section{IEEE 802.15.4}

Driven by an IEEE working group the 802.15.4 standard provides the physical and
mac access control layer for so called low-rate wireless personal area network
(LR-WPAN). The emphasis is on low cost devices for communication without
infrastructure in a nearby environment. The communication range would be up to
10m and offers nowadays a transfer rate up to 250 kbit/s. The standard specifies
three possible frequency bands to operate in. The usage of some may be
restricted in different countries but one band is available worldwide. Other
features of 802.15.4 includes collision avoidance through CSMA/CA, build in
support for secure communication through cryptography, power management through
link quality control and energy detection as well as reserved time slots for
real-time operations.

As already covered the standard does only cover the two lowest layer of a
protocol stack. Supplement it to a fully functional networking stack is the aim
of different other specifications. The most famous would be ZigBee. With 6LowPAN
there is also work underway to link LR-WPAN together with standard internet
protocols like IPv6.

Specified for an infrastructure-less network the topologies may be star or
Peer-to-Peer based as well a composition of both. Two different device types are
allowed. full-function device (FFD) and reduced-function device (RFD). The RFD
is a really simple device which can only connect to one FFD at a time. Therefor
it can only act as a leaf in all described topologies. In contrast the FFD is
able to act as a coordinator to span up a whole PAN and rely messages to other
nodes. At least one coordinator is needed in every network. One thing to keep in
mind is that routing is not covered by the standard. To rely messages over
different hops the supplement upper layers need to take care of this.

\section{DTN}

Delay-tolerant networking is the other core technology that is used within this
thesis. The need for communication between nodes without continuous network
connectivity is what DTN seeks to address. Most modern routing protocols only
send out the real data once a complete route to the destination is established.
A communication over such an approach is only possible if source and destination
are long enough within a connected network to establish a route between them,
transfer the data and maybe acknowledge the transfer.

DTN in contrast uses a store and forward approach which sends out the data
directly together with the destination address. The bundle protocol in
\cite{RFC5050} was specified for this. One approach to maximize the propability
of a successfull delivered message would be to send out multiple copies of the
same message, maybe to different hops. Such an approach obviously increases the
network and storage load and may not be usefull for constrained devices like
sensor nodes.

The bundle protocol specifies an overlay network which interacts with the lower
layers over bundle convergence layers. The lower layers are not bound to be IP
based even if that is the one widely used. In this thesis we describe the usage
of the DTN implementation IBR-DTN over a 802.15.4 radio link on a linux based
system.


\chapter{iMote2 Hardware}
\section{Base board}
\section{CC2420 802.15.4 Transceiver}


\chapter{Base System}
\section{Bootstraping}
\section{Linux Kernel}
\section{CC2420 Driver and ieee802154 stack}
\section{OpenEmbedded base system}


\chapter{IBR-DTN Implementation}
\section{IBR-DTN}

IBR-DTN is an implementation of the DTN bundle protocol. It is designed and
implemented with efficiency in mind. Efficient memory usage is critical for
use in embedded systems like wireless routers or smartphones. Another design
goal is the interoperability with DTN2 the reference implementation by the Delay
Tolerant Networking Research Group (DTNRG).

The modular implementation comprise a DTN Core, Convergence LAyer Manager,
Bundle Router and Persistent storage.

\section{Other DTN Implementations}

DTN2, ION, DTNLite on TinyOS, DASN for symbian smartphones


\chapter{802.15.4 Convergence layer}
\label{802154layer}
\section{Interface with the linux ieee802154 Stack}
\section{Addressing}
\section{Fragmentation}
\section{Related Work}

While DTN is getting more popular for research there are only a few research
projects then combine DTN with sensor network technologies like IEEE 802.15.4.
That is not surprising if one keeps in mind that the DTN architecture is is
complex and memory as well as bandwidth consuming. Both resources that are scare
on sensor network platforms.

Two projects are aiming to explore this area of research. One of them is the
TinyOS based DTNLite \cite{dtnlite}. It is loosely based on the DTN overlay
architecture. To deliver custody transfer  over the overlay they use techniques
like asynchronous message delivery. The concept is targeted at data collection
applications which need to reach a central station for the collection over the
network.

Contiki DTN



\chapter{Evaluation}
\subsection{General}
\subsection{Performance}
\subsection{Battery Life}


\chapter{Conclusion}

The transmission of small DTN bundles is possible over 802.15.4. Biggest
restrictions are the small payload of only 40 Byte per bundle due to missing
fragmentation in smaller chunks and the slow throughput on the 802.15.4 layer.
With only 575 Bytes per second this is only 0.22 percent of the theoretical
maximum of 250 Kbyte per second. More work on the driver and the ieee802154
respectively the mac802154 stack are needed to avoid the 100ms sleep in
userspace.

On the IBR-DTN side it needs also to be investigated why sometimes the bundle
transmission stalls for several seconds and continues normally afterwards. It
could be the interaction of IBR-DTN with the socket interface to the kernel as well
as some parts in IBR-DTN that needs performance optimizations.


\addcontentsline{toc}{chapter}{Literature}
\bibliographystyle{unsrt}
\bibliography{thesis}

\begin{appendix}

Use OpenOCD to do the initial blob install:

The iMotes are shipped with something like TinyOS by default. I wasn't able to
find a bootloader on it so the only way to get it bootstrapped was to use the
JTAG interface.

\begin{verbatim}
Get OpenOCD (v0.4.0) and libftdi (v0.17)
Download the pre-compiled blob from: http://sourceforge.net/projects/imote2-linux/files/
(As root or with fixed US permissions) sudo openocd -f /usr/share/openocd/scripts/interface/jtagkey.cfg -f /usr/share/openocd/scripts/board/crossbow_tech_imote2.cfg
On another console: telnet localhost 4444
# reset halt
# flash protect 0 0 last off
# flash erase_sector 0 0 258
# flash write_image blob-im2
# flash write_image 2.6.34-rc2-zImage 0x00040000 bin
# flash write_image console-image-imote2.jffs2 0x00240000 bin
\end{verbatim}

\chapter{OpenEmbedded}
\label{annexoe}

OpenEmbedded~\footnote{http://www.openembedded.org/} is a build framework for
embedded Linux. It provides the meta-data which incorporates the knowledge how
to fetch, build, install, package and more for over 5000 Open Source projects.

Adding support for the Imote2 hardware to it did only required us to add a
\emph{machine configuration} which describes the hardware and let OpenEmbedded
know about it. Below is this machine configuration we added:

\begin{verbatim}
#@TYPE: Machine
#@Name: Crossbow iMote2
#@DESCRIPTION: Machine configuration for Crossbow iMote 2
TARGET_ARCH = "arm"
PREFERRED_PROVIDER_virtual/kernel = "linux"
PACKAGE_EXTRA_ARCHS = " iwmmxt"
KERNEL_IMAGETYPE = "zImage"
IMAGE_FSTYPE += "jffs2"
EXTRA_IMAGECMD_jffs2 = "--l --pad=0x01DC0000 --eraseblock=0x20000"
CMDLINE="root=/dev/mtdblock2 rootfstype=jffs2 console=ttyS2,115200"
SERIAL_CONSOLE = "115200 ttyS2"
require conf/machine/include/tune-xscale.inc
ROOT_FLASH_SIZE ?= "30"
MACHINE_FEATURES = "kernel26 usbgadget alsa iwmmxt"
\end{verbatim}

We also added the needed meta-data for the Linux kernel used on our Imote2 and
the build description for IBR-DTN. In a final step we put all this together in
an image that could be build with the \emph{bitbake} application. BitBake uses
the meta-data provided by OpenEmbedded to actually build the software and the
resulting firmware and kernel images.

After this we were able to create our complete firmware image by just executing
BitBake like this:

\begin{verbatim}
bitbake imote2-image
\end{verbatim}

All our changes are already included in the OpenEmbedded source code repository
and could be used from there. The latest git revision we used for our work was
6fc1c96131d4c4afabe815148f8e55656fb7f402. If any newer version have problems it
is possible to revert back to this known good revision for future work.


\end{appendix}

\end{document}
